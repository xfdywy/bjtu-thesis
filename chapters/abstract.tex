\begin{abstract}
成像技术在军事、工业、民用等领域具有广泛的应用。然而,待测场景与时
延成像系统之间存在相对运动,气动光学效应等原因都会造成图像随机运动模
糊,影响机器视觉系统的成像性能。图像复原算法是一种可以改善图像质量的
有效手段,然而,应用最为广泛的盲复原算法因采用了迭代运算而存在运算量
大、耗时长的缺点,难以满足各领域对退化图像快速复原的技术要求,而线性
复原算法因现有像质评价函数或方法的各种局限性而无法实际应用。因此,建
立一种新的像质评价模型,将其与线性复原算法相结合,是实现随机运动模糊
图像快速复原的有效手段。

本课题“基于动态光学传递函数的随机运动模糊图像快速复原方法”的
研究目的在于:严格推导随机运动动态光学传递函数(Dynamic Optical Transfer
Function,DOTF)模型,将其用于准确描述随机运动模糊图像的像质退化规律,
提出一种从退化图像中提取DOTF,并将其与线性复原算法相结合的图像复原算
法,从而回避迭代运算,实现随机运动模糊图像快速复原。本课题的研究为随
机运动模糊图像提供一种快速复原手段,研究成果对于提高时延成像系统实时
成像性能具有积极的促进作用。
课题的主要研究内容有:

\begin{enumerate}
\item 为了解决现有像质评价函数或方法不能用于指导图像快速复原,建立
了一个新的像质评价函数。首先分别在空域和频域严格推导了随机运动DOTF 模
型,根据两种方法得到的DOTF 模型相同,实现了二者之间的相互验证;然后在
静止、匀速运动和高频简谐振动三种典型运动状态下,将本文模型化简为现有
模型形式,验证了本文DOTF模型适用于随机运动;最后通过严格的数学推导,
发现本文随机运动DOTF 模型是气动光学点扩展函数的傅里叶变换,说明本文随
机运动DOTF 模型可用于描述气动光学随机运动模糊图像像质退化规律。

\item 针对盲复原算法运算量大,运算时间长的问题,提出了一种随机运动
模糊图像快速复原方法。首先推导了随机运动模糊图像的频谱模型,然后利用
退化图像向导部分在背景均匀条件下所具有的边界补充特性和背景等效静止特
性,提出了一种从退化图像向导部分提取DOTF,并与线性滤波复原方法相结合
的复原方法,最后对该复原方法进行仿真,仿真结果表明,复原图像与原始图
像之间的相关系数为1,验证了该方法的有效性。

\item 分析了本文方法中各项误差因素对图像复原质量的影响规律。针对背
景不均匀,图像传感器存在暗电流,以及伴随有随机噪声的三种情况,分别建
立了随机运动模糊图像复原模型并仿真,仿真结果分别为:对于背景不均匀的
情况,复原图像与原始图像的相关系数随背景对比度的增大而变小;对于图像
传感器存在恒量暗电流的情况,只要在复原环节将暗电流从退化图像中减去,
复原图像与原始图像的相关系数就恒为1;对于伴随有随机噪声的情况下,复原
图像与原始图像的相关系数随噪声强度的增加而减小。这些结果表明为了获得
更好的图像复原效果,需要尽可能满足背景均匀条件,降低噪声强度,并在复
原环节将暗电流从退化图像中减去。

\item 对本文图像快速复原算法及相关理论进行了实验验证。首先根据相
同运动参数下的DOTF测量曲线与理论曲线吻合,以及DOTF测量曲线随速度
变化而在空间频率方向表现出的伸缩规律与理论分析相一致,验证了本文随
机运动DOTF模型的正确性;然后根据软件得到的退化图像与实际采集得到的
退化图像之间的相关系数为0.9979,验证了向导部分的边界补充特性,根据背
景随目标一起运动与不随目标一起运动情况下得到两幅退化图像之间的相关
系数为0.9994,验证了向导部分的背景等效静止特性;最后分别采用文本复原
方法与现有快速复原方法复原同一幅退化图像,结果为:表征图像复原质量
的GMG和LS,本文方法均高于现有方法5.77倍以上,而复原时间却减少了将
近90\%,结果表明本文方法不仅具有更好的复原效果,而且将运算速度提高近一
个数量级
\end{enumerate}
	
\keywords{图像复原;DOTF;随机运动;快速算法}
\end{abstract}