\chapter[附录标题]{}
%	\addcontentsline{toc}{section}{附录\thechapter\hspace*{1em}附录标题}
\markboth{附录\thechapter}{}
\begin{center}
	\zihao{3}\bfseries 附录标题
\end{center}

[内容为五号宋体。] 附录是作为论文主体的补充项目,并不是必须的。
论文的附录依序用大写正体英文字母A、B、C……编序号,如:附录A。


%正文中加入index:
%This is my key\index{key}.
This is my second palace\index{分类索引!palace} that has a key.

上文提到Knuth留下了后门 \verb|\special|,但是直接用它来插入图形不够含蓄优雅,于是 \LaTeX v2.09 推出了\texttt{epsf} 和 \texttt{psfig} 宏包。之后David P. Carlisle (1961--) \index{著者索引!Carlisle@David P. Carlisle, 大卫·卡利斯}\footnote{1995年曼彻斯特大学\index{关键词索引!edu.manchester@University of Manchester, 曼彻斯特大学}数学博士,剑桥博士后,1998年加入数字算法公司 (Numerical Algorithms Group) \index{关键词索引!com.nag@Numerical Algorithms Group, 数字算法公司}。} 和Rahtz推出了面向 \LaTeXe 的 \texttt{graphics} 和 \texttt{graphicx} 宏包;后者基于前者,语法更简单,功能更强大,所以一般推荐用它。


\index{分类索引!plus}