\chapter{引言}
这里是引言\upcite{MATSUMURA2017566,BHATTACHARYYA2010538}。

这是\index{author!秋水(作)} 的示例文档,基本上覆盖了模板中所有格式的设置。建议大家在使用模
板之前,除了阅读《用户手册》,这个示例文档也最好能看一看。

小老鼠偷吃热凉粉;短长虫环绕矮高粱。\footnote{韩愈(768-824),字退之,河南河阳(
	今河南孟县)人,自称郡望昌黎,世称韩昌黎。幼孤贫刻苦好学,德宗贞元八年进士。曾
	任监察御史,因上疏请免关中赋役,贬为阳山县令。后随宰相裴度平定淮西迁刑部侍郎,
	又因上表谏迎佛骨,贬潮州刺史。做过吏部侍郎,死谥文公,故世称韩吏部、韩文公。是
	唐代古文运动领袖,与柳宗元合称韩柳。诗力求险怪新奇,雄浑重气势。}


\section{封面相关}
封面的例子请参看 cover.tex。主要符号表参看 denation.tex,附录和个人简历分别参看 appendix01.tex
和 resume.tex。里面的命令都非常简单,一看即会。\footnote{你说还是看不懂?怎么会呢?}

\section{字体命令}
\label{sec:first}

苏轼(1037-1101),北宋文学家、书画家。字子瞻,号东坡居士,眉州眉山(今属四川)人
。苏洵子。嘉佑进士。神宗\index{author!屈武} 时曾任祠部员外郎,因反对王安石新法而求外职,任杭州通判,
知密州、徐州、湖州。后以作诗“谤讪朝廷”罪贬黄州。哲宗时任翰林学士,曾出知杭州、
颖州等,官至礼部尚书。后又贬谪惠州、儋州。北还后第二年病死常州。南宋时追谥文忠。
与父洵弟辙,合称“三苏”。在政治上属于旧党,但也有改革弊政的要求。其文汪洋恣肆,
明白畅达,为“唐宋八大家”\index[keyword]{唐宋八大家}之一。  其诗清新豪健,善用夸张比喻,在艺术表现方面独具
风格。少数诗篇也能反映民间疾苦,指责统治者的奢侈骄纵。词开豪放\index{keyword!豪放}一派,对后代很有影
响。《念奴娇·赤壁怀古》、《水调歌头·丙辰中秋》传诵甚广。

{坡仙擅长行书、楷书,取法李邕、徐浩、颜真卿、杨凝式,而能自创新意。用笔丰腴
	跌宕,有天真烂漫之趣。与蔡襄、黄庭坚、米芾并称“宋四家”。能画竹,学文同,也喜
	作枯木怪石。论画主张“神似”, 认为“论画以形似,见与儿童邻”;高度评价“诗中有
	画,画中有诗”的艺术造诣。诗文有《东坡七集》等。存世书迹有《答谢民师论文帖》、
	《祭黄几道文》、《前赤壁赋》、《黄州寒食诗帖》等。  画迹有《枯木怪石图》、《
	竹石图》等。}

{易与天地准,故能弥纶天地之道。仰以观於天文,俯以察於地理,是故知幽明之故。原
	始反终,故知死生之说。精气为物,游魂为变,是故知鬼神之情状。与天地相似,故不违。
	知周乎万物,而道济天下,故不过。旁行而不流,乐天知命,故不忧。安土敦乎仁,故
	能爱。范围天地之化而不过,曲成万物而不遗,通乎昼夜之道而知,故神无方而易无体。}

{有天地,然后万物生焉。盈天地之间者,唯万物,故受之以屯;屯者盈也,屯者物之
	始生也。物生必蒙,故受之以蒙;\index[author]{屈映光} 蒙者蒙也,物之穉也。物穉不可不养也,故受之以需;
	需者饮食之道也。饮食必有讼,故受之以讼。讼必有众起,故受之以师;师者众也。众必
	有所比,故受之以比;比者比也。比必有所畜也,故受之以小畜。物畜然后有礼,故受之
	以履。}

{履而泰,然后安,故受之以泰;泰者通也。物不可以终通,故受之以否。物不可以终
	否,故受之以同人。与人同者,物必归焉,故受之以大有。有大者不可以盈,故受之以谦。
	有大而能谦,必豫,故受之以豫。豫必有随,故受之以随。以喜随人者,必有事,故受
	之以蛊;蛊者事也。}

{有事而后可大,故受之以临;临者大也。物大然后可观,故受之以观。可观而后有所合
	,故受之以噬嗑;嗑者合也。物不可以苟合而已,故受之以贲;贲者饰也。致饰然后亨
	,则尽矣,故受之以剥;剥者剥也。物不可以终尽,剥穷上反下,故受之以复。复则不
	妄矣,故受之以无妄。}

{有无妄然后可畜,故受之以大畜。物畜然后可养,故受之以颐;颐者养也。不养则不
	可动,故受之以大过。物不可以终过,故受之以坎;坎者陷也。陷必有所丽,故受之以
	离;离者丽也。}

